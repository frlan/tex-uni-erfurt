\documentclass[
	a4paper,%
	oneside,%
	12pt,%
	halfparskip,%
	headinclude,%
	headsepline,% Trennungslinie fuer Kopfzeile
	plainheadsepline,
	footsepline, % Trennungslinie fuer Fußzeile
	plainfootsepline,
	bibtotoc,%
	liststotoc,%
	toc=bibliography,
	]{scrbook}

\date{\today}

\usepackage[
	a4paper,
	left=3cm,
	right=3cm,
	top=3cm,
	bottom=3cm
	]{geometry}


% 1,5 Zeilen Text
\usepackage[onehalfspacing]{setspace}

% Schrift und Input-Encoding
\usepackage{palatino}
\usepackage[T1]{fontenc}
\usepackage[utf8]{inputenc}
\usepackage[draft=false]{microtype}

% Deutsche Bezeichnungen für Inhaltsverzeichnis etc.
\usepackage[german]{babel}

% lange Tabelle
\usepackage{tabularx}

% Eigenschaften des PDF
\usepackage[%
	pdfborder=0,%
	colorlinks=false,%
	linkcolor=black,%
	menucolor=black,%
	urlcolor=black,%
	pdftitle={Arbeitstitel},%
	pdfauthor={Name},%
	pdfsubject={Meine Masterarbeit von mir}
	]{hyperref}

% Kleiner Hack für Versionskompatibilität
\usepackage{fixltx2e}

% Fußnoten nicht pro Kapitel sondern für komplette Arbeit zählen
\usepackage{remreset}
\makeatletter
	\@removefromreset{footnote}{chapter}
\makeatother

% Tiefe der Nummerierung für Abschnitte und Inhaltsverzeichnis festlegen
\setcounter{secnumdepth}4{}
\setcounter{tocdepth}{3}

% Quellenverwaltung
% Alternativen:
%%%%%%%%%%%%%%%%%%%%%%%%%%
% 1. Biblatex
% Einige Stile gibt es unter
% http://wiki.lyx.org/BibTeX/Biblatex#toc3
%\usepackage[style=authoryear-comp]{biblatex}
%\usepackage[style=authoryear-comp,sorting=nty]{biblatex}
%\usepackage[babel,german=guillemets]{csquotes}
%\bibliography{quellen.bib}
%%%%%%%%%%%%%%%%%%%%%%%%%%%
% 2. bibtex + apacite
% zum Erstellen von APA-kompatiblen Quellenverzeichnissen und Verweissen
\usepackage[hyper]{apacite}
\bibliographystyle{apacite}

% Abkürzungsverwaltung
% Dokumentation: http://de.wikibooks.org/wiki/LaTeX-W%C3%B6rterbuch:_Abk%C3%BCrzungsverzeichnis
\usepackage[nohyperlinks,printonlyused]{acronym}


% Einbinden von Bilder und ggf. PDF
\usepackage[final]{graphicx}
\usepackage{pdfpages}

% TODO
\usepackage[obeyDraft,
	ngerman,
	shadow,
	textsize=tiny,
	backgroundcolor=white,
	linecolor=black
]{todonotes}

% Noch ein paar zusätzliche Pakete:
% Zu Ausgabe von WWW-Adressen und Emailadressen
\usepackage{url}

% Ein paar weitere math. Symbole
\usepackage{amsmath}
\usepackage{amssymb}

% Schön formatierte Brüche ala ¼
\usepackage{nicefrac}

% Das €-Symbol
\usepackage{eurosym}

% Freibewegliche Graphiken und Tabellen
\usepackage{float}

% Optimierung von Tabellen-Linien auf Buch
\usepackage{booktabs}

% ISO-konforme gr. Buchstaben
\usepackage{fixmath}

% Zeilennummern
\usepackage{lineno}

% Aus Faulheit wird ein Befehl \vgl{} erstellt, der eine Fußnote mit
% Querverweis auf anderes Kapitel erstellt (-> \label{}
\newcommand*{\vgl}[1]{\footnote{Vgl. dazu die Ausführungen unter %
    Kapitel \ref{#1}}}

% Frage und Antwort
\newcommand{\frage}[1]{{\sc Frage:} \par #1 \par}
\newcommand{\antwort}[1]{\noindent{\sc Antwort:} \par {#1} }


% Das Dokument beginnt
\begin{document}

%%%%%%%%%%%%%%%%%%%%%%%%%%%%%%%%%%%%%%%%%%%%%%%%%%%%%%%%%%%%%%%%%%%%%%55
% Zusammensetzen der Titelseite
%%%%%%%%%%%%%%%%%%%%%%%%%%%%%%%%%%%%%%%%%%%%%%%%%%%%%%%%%%%%%%%%%%%%%%55

% Da Template der Uni Erfurt bisschen anders mag, wird die Seite selber
% zusammen gesetzt.

\begin{titlepage}
 \thispagestyle{empty}
  %Institution, an der die Arbeit geschrieben wurde
  \includegraphics[width=0.30\textwidth]{pics/Logo_Universitaet_Erfurt.png}\\
	Universität Erfurt \\
	Fakultät \\
	Studiengang

	%dynamischer Zwischenraum bis Seite voll ist
	\vfill{}

	\begin{center}
			\Huge\textbf{Arbeitstyp}
	\end{center}

	\begin{center}
			\Huge\textbf{Thema} \\
			\Large Untertitel
	\end{center}

	%dynamischer Zwischenraum bis Seite voll ist
	\vfill{}

%HowTo für Tabulatoren:
%http://cornitex.wiki-site.com/index.php/Tabulatoren_in_LaTeX_-_Tabbing
% \> Tabulator, linksbündig
% \+ Einzug in den nächsten Zeilen um diesen Tabulator erhöhen
% \- Einzug in den nächsten Zeilen um diesen Tabulator verringern
\begin{tabbing}
	Vorgelegt von:\hspace{0.5em} \= Matrikelnummer\hspace{0.7em} \=  Mustermannstra0e \kill
	Vorgelegt von: \+\> Name\\
		Anschrift: \+\> Straße \\
		\-Ort \\
	Martikelnummer: \> Martikelnummer \\
	\-Email: \> Email \\[2ex]
	Betreut durch: \+\> Erstgutachter \> Prof. Dr. XXX \\
						Zweitgutachterin: \> Prof. Dr. XXX \\
\end{tabbing}

\vspace{5ex}

% Datum auf Titelseite: Auf Abgabgedatum setzen
\today{}

\end{titlepage}

\listoftodos{}

%%%%%%%%%%%%%%%%%%%%%%%%%%%%%%%%%%%%%%%%%%%%%%%%%%%%%%%%%%%%%%%%%%%%%%55
% Titelseite zuende
%%%%%%%%%%%%%%%%%%%%%%%%%%%%%%%%%%%%%%%%%%%%%%%%%%%%%%%%%%%%%%%%%%%%%%55

%%%%%%%%%%%%%%%%%%%%%%%%%%%%%%%%%%%%%%%%%%%%%%%%%%%%%%%%%%%%%%%%%%%%%%55
% Eigentliches Dokument
%%%%%%%%%%%%%%%%%%%%%%%%%%%%%%%%%%%%%%%%%%%%%%%%%%%%%%%%%%%%%%%%%%%%%%55

% Vorspann des Dokumentes
\frontmatter{}
\tableofcontents{}
%\listoffigures{}
%\listoftables{}

%Hauptteil
\mainmatter{}

% Einleitung auf 0. setzen
\addtocounter{chapter}{-1}


\chapter{Einleitung}
\label{sec:einleitung}


% Text
% Hier normal \chapter, \section ... verwenden


% Anhang

\appendix{}

\backmatter{}

% Sollten durch eigenltiches BibTeX-file ersetzt werden
\bibliography{quellen}
\chapter{Eigenständigkeitserklärung}

Hiermit versichere ich, dass ich die vorliegende Magisterarbeit
selbstständig verfasst und keine anderen als die angegebenen Quellen
und Hilfsmittel benutzt habe, alle Ausführungen, die anderen Schriften
wörtlich oder sinngemäß entnommen wurden, kenntlich gemacht sind und
die Arbeit in gleicher oder ähnlicher Fassung noch nicht Bestandteil
einer Studien- oder Prüfungsleistung war.

\hspace{5ex}

Jena, den DATUM \\[7ex]
Name


\end{document}
